\documentclass[a4paper,fleqn,usenatbib]{mnras}
\usepackage[T1]{fontenc}
\usepackage{ae,aecompl}


\usepackage{graphicx}	% Including figure files
\usepackage{amsmath}	% Advanced maths commands
\usepackage{amssymb}	% Extra maths symbols
\usepackage{subfig}
\usepackage{array}


\usepackage{multirow}
\usepackage{multicol}
\usepackage{blindtext}
\newcolumntype{?}{!{\vrule width 1pt}}
\newcommand{\Msun}{\,{\rm Mpc}$_{\odot}$\,}
\newcommand{\Mpch}{\,{\rm Mpc}\,\ifmmode h^{-1}\else $h^{-1}$\fi}
\newcommand{\kpch}{\,{\rm kpc}\,\ifmmode h^{-1}\else $h^{-1}$\fi}
\newcommand{\kpc}{\,{\rm kpc}\,}


\newenvironment{Table}
   {\par\bigskip\noindent\minipage{\columnwidth}\centering}
   {\endminipage\par\bigskip}

\title[The shape of dark matter haloes in the Auriga simulations]
{Dark matter halo shape in the Auriga simulations: radial
  dependence, time evolution and correlation with baryon disk
  properties}
\author[Jesus Prada,  Jaime E. Forero-Romero, Volker Springel ]{
Jesus Prada,$^{1}$\thanks{E-mail: jd.prada1760@uniandes.edu.co}
Jaime E. Forero-Romero,$^{1}$
Volker Springel$^{2}$
\\
% List of institutions
$^{1}$Departamento de F\'sica, Universidad de los Andes, Cra. 1 No.
18A-10, Edificio Ip, Bogot\'a, Colombia.\\
$^{2}$Heidelberg Institute for Theoretical Studies, Schloss-Wolfsbrunnenweg 35, D-69118 Heidelberg
Germany.\\
}

% These dates will be filled out by the publisher
\date{Accepted XXX. Received YYY; in original form ZZZ}

% Enter the current year, for the copyright statements etc.
\pubyear{2018}

% Don't change these lines
\begin{document}
\label{firstpage}
\pagerange{\pageref{firstpage}--\pageref{lastpage}}
\maketitle

% Abstract of the paper
\begin{abstract}
We present shape measurements of dark matter halos in a suit
of 30 cosmological simulations from the Auriga project.
For all measurements we compare the results with full
magnetohydrodynamics and dark matter only physics.
We find a strong influence of baryons in making dark matter halos rounder at all
radii compared to its dark matter only counterparts.
This change in triaxility its more pronounced towards the inner
regions of the halo.
At distances close to the disk rounder dark matter distributions
correlate with massive stellar disks and low gas density.
The dark matter shape shows a strong alignment with the
stellar disk shape as a function of radius, the stronger aligmnet is
found at $~50$kpc, this alignment weakens at smaller and larger radii. 
In dark matter only simulations the time evolution of the shape
measured at the virial radius correlates with the radial shape
evolution at present day.  
However, this correlation is lost once the baryonic effects are
included. 
We conclude that in our simulation suite that baryonic physics
produces rounder halos with a strong correlation between its inner
shape and the stellar and gaseous properties of the disk.
\end{abstract}

% Select between one and six entries from the list of approved keywords.
% Don't make up new ones.
\begin{keywords}
galaxies: evolution --- galaxies: formation --- galaxies: haloes ---
dark matter
\end{keywords}

%%%%%%%%%%%%%%%%%%%%%%%%%%%%%%%%%%%%%%%%%%%%%%%%%%

%%%%%%%%%%%%%%%%% BODY OF PAPER %%%%%%%%%%%%%%%%%%

\section{Introduction}


A robust prediction of the Cold Dark Matter (CDM) paradigm is that DM
halos are ellipsoidal and can be characterized by the principal axes
$a>b>c$.
This ellipsoidal shape is mostly due to the anisotropical and
clumpy accretion of matter influenced by environmental structures.
Numerical studies how that the shape has a strong mass dependence
\citep{Allgood_et_al._2006}, halos are also rounder at the outerskirts
than at the inner part. 
Shape also evolves with cosmic time, halos get
rounder as they evolve.  

There is however a high degree of uncertainty on what is the degree of
uncertainty on the degree of ellipticity of the Milky Way DM halo.
This problem has been addressed both by observations and simulations.
The difficulty in making an observational measurement lies in the
indirect nature of the effect; i.e. the ellipticity can only be
constrained by its effects on quantities such as stellar radial
velocities.
In simulations the uncertainty on predicting the MW DM ellipticity is 
driven by the different physical effects that should be modeled and
its different possible numerical implementations.


Observationally some studies prefer oblate (i.e. a=b>c) configurations at small
distances around $\leq 20$ kpc
\citep[see][]{Law_and_Majewski_2010,Bovy_et_el._2016,Loebman_et_al._2012,Olling_and_Merrifield_2000,Banerjee_and_Chanda_2011} 
and more triaxial and prolate configurations on the outter distances
$\geq 20$ kpc 
\citep[see][]{Vera-Ciro_and_Helmi_2013,Law_and_Majewski_2009,Deg_and_Widrow_2013,Banerjee_and_Chanda_2011}.
However, some  studies are inclined towards prolate configurations even at the inner
parts of the halo \citep[see][]{Bowden_et_al._2016}, and
although it previously seemed that a triaxial DM halo on the
outerskirts would be necessary to fully explain the characterization
of the Sagittarius stream \citep{Law_and_Majewski_2009}, recent studies
questioned this claim by reporting inconsistencies with narrow stellar
streams \citet{Pearson_et_al._2015} or finding that
the relaxation of other constraints may make this claim unnecessary
\citet{Ibata_et_al._2013}. 

In simulations there is strong evidence claiming that the presence of
baryons produces axisymmetrical halos.  
For instance, some studies have shown that the DM halo shape must be
axisymmetrical to ensure the stability of a hydrodynamical disk
embeded in a static DM halo. 
Other have studied this rounding effect by simulating the disk as rigid
potential inside an N-body triaxial DM
halo \cite{Debattista_et_al._2008,Debattista_et_al._2013,Kazantzidis_et_al._2010}
finding that the halo responds to the disk by becoming less triaxial. 

The caveat of the studies mentioned above is that they do not
follow baryons in the whole cosmological context. 
Other studies overcome this limitation by using resimulations 
\citep{Abadi_et_al._2010,Bryan_et_al._2013} finding that the
feeback related to star formation in the disk drives the strenght of
the round effect. 
Recently \cite{2018arXiv180907255C} made a study in a cosmological
simulation to compare the effect of including baryons. They do find,
on average, rounder halo shapes once hydrodynamic effects are
included, but it is uncertain the strenght of this statistical effect
on galaxies similar to the MW.


All these difficulties (enough numerical resolution, explicit
cosmological context, appropriate feedback physics to produce
realistic MW disks) have limited the studies that want to study the
rounding effect of baryons in MW-like galaxies.
In this work we overcome all these limitations by analyzing the
results of state-of-the-art hydrodynamical simulations of isolated
halos that resemble the Milky Way.
We also perform a convergence study with simulation performed at
different resolution levels and explicitly compare the role of DM only
vs. DM+hydro on the MW DM halo shape.

\begin{table*}
\centering
\begin{tabular}{l|cc|cc|cc|cc}
\hline
\hline
Halo & \multicolumn{2}{c}{$N_P$/$10^6$} & \multicolumn{2}{c}{$M_P$/$10^5M_\odot$} & 
\multicolumn{2}{c}{$R_{vir}$/kpc} & \multicolumn{2}{c}{$M_{vir}$/$10^{14} M_\odot$}  \\ \hline
& DM & MHD& DM & MHD& DM & MHD& DM & MHD\\ \hline \hline
halo 1&4.068&2.447&2.397&2.022&196.927&187.674&9.062&7.844\\
halo 2&5.625&5.457&2.481&2.093&235.094&233.934&15.418&15.191\\
halo 3&3.826&3.852&2.645&2.231&210.693&210.955&11.099&11.141\\
halo 4&4.585&4.530&2.590&2.185&219.378&215.438&12.529&11.866\\
halo 5&3.262&3.290&2.533&2.137&196.984&197.246&9.071&9.106\\ 
halo 6 ($\star$)&3.184&3.110&2.337&1.972&191.840&189.342&8.378&8.054\\ 
halo 7&3.878&3.729&2.296&1.937&197.864&196.509&9.193&9.005\\
halo 8&2.772&2.796&2.451&2.068&190.716&191.764&8.231&8.368\\
halo 9&3.038&3.010&2.738&2.310&195.826&190.640&8.911&8.222\\
halo 10&2.700&2.751&2.541&2.144&187.139&188.147&7.777&7.904\\
halo 11&4.146&4.116&2.541&2.144&221.821&219.568&12.952&12.560\\
halo 12&2.865&2.908&2.645&2.231&192.280&192.038&8.436&8.404\\
halo 13&3.520&3.600&2.393&2.019&202.139&203.815&9.801&10.048\\
halo 14&4.200&4.475&2.499&2.108&215.535&218.927&11.882&12.453\\
halo 15&2.888&2.845&2.541&2.144&199.848&200.658&9.471&9.588\\ 
halo 16 ($\star$)&3.821&3.871&2.499&2.108&212.590&212.632&11.401&11.408\\ 
halo 17&2.752&2.781&2.552&2.153&188.067&187.404&7.893&7.811\\
halo 18&3.770&3.624&2.738&2.310&201.124&207.293&9.655&10.571\\
halo 19&2.989&3.086&2.645&2.231&200.244&200.325&9.527&9.540\\
halo 20&3.903&3.822&2.481&2.093&210.097&211.423&11.005&11.214\\ 
halo 21 ($\star$) &4.105&4.075&2.640&2.227&219.527&219.823&12.555&12.604\\ 
halo 22&2.794&2.766&2.625&2.215&188.363&184.801&7.931&7.489\\ 
halo 23 ($\star$) &3.977&4.073&2.795&2.358&217.768&215.959&12.254&11.952\\
halo 24 ($\star$) &4.466&4.426&2.522&2.127&217.440&215.147&12.199&11.817\\ 
halo 25&2.902&2.806&2.645&2.231&199.922&198.299&9.482&9.254\\
halo 26 &4.610&4.716&2.506&2.115&219.984&218.939&12.633&12.454\\
halo 27 ($\star$) & 5.060&5.018&2.590&2.185&228.036&226.225&14.071&13.740\\ 
halo 28 & 4.184&4.276&2.645&2.231&216.979&217.997&12.121&12.294\\
halo 29 & 4.827&4.613&2.499&2.108&225.791&219.935&13.660&12.625\\
halo 30 & 3.268&3.112&2.579&2.176&195.043&194.741&8.805&8.763\\
\hline
\hline
\end{tabular}
\caption{Specifications of each level 4 galaxy (halo). 
  The DM and MHD versions of each parameters are presented together. 
  The columns correspond to: (1) Halo name, (2,3) Millions of DM
  particles belonging to the halo, (4,5) DM particle mass in
  $10^5M_\odot$, (6,7) Halo Virial radius in kpc and (8,9) halo virial
  mass in $10^{14}M_\odot$. Halos marked with a star ($\star$) are
  correspond to halos resimulated at higher resolution (level 3).}  
\label{tab:level4}
\end{table*} 

\begin{table*}
\centering
\begin{tabular}{l?cc?cc?cc?cc}
\hline
\hline
Halo & \multicolumn{2}{c?}{$N_P$/$10^6$} & \multicolumn{2}{c?}{$M_P$/$10^5M_\odot$} & \multicolumn{2}{c?}{$R_{vir}$/kpc}&\multicolumn{2}{c}{$M_{vir}$/$10^{14}M_\odot$}  \\ \hline
& DM & MHD& DM & MHD& DM & MHD& DM & MHD\\ \hline \hline
halo 6&24.902&24.185&0.292&0.246&191.741&188.367&8.365&7.932\\
halo 16&29.750&30.334&0.312&0.263&212.622&212.542&11.406&11.395\\
halo 21&31.993&31.503&0.330&0.278&219.731&220.250&12.588&12.679\\
halo 23&31.379&31.618&0.349&0.295&217.793&213.358&12.259&11.524\\
halo 24&34.987&35.153&0.315&0.266&217.313&213.963&12.179&11.624\\
halo 27&39.617&39.056&0.324&0.273&227.908&223.484&14.048&13.244\\
\hline
\hline
\end{tabular}
\caption{Same layout Table \ref{tab:level3} for Level 3 simulations (higher
  resolution than Level 4 simulations).}
\label{tab:level3}
\end{table*} 



\section{Numerical Simulations}


The Auriga project offers cosmological zoom in simulations of MW-sized 
dark matter halos in a $\Lambda$CDM cosmology. 
This simulations come in two versions: dark matter only and
baryonic physics including magetohydrodynamics (MHD).
A detailed description of the simulations and their
disk properties can be founc in \citep{auriga}, here we summarize its
main features.

The objects in those simulations were selected from a set of 30
isolated halos in the Evolution and Assembly of GaLaxies and their
Environments (EAGLE)  project \citep{Eagle}.   
These halos were randomly selected from a sample of the most isolated
halos at $z=0$ whose virial mass $M_{200}$ was between $10^{12}M_\odot$ and
$2\times 10^{12}M_\odot$. 
The cosmological parameters in these simulatins correspond to
$\Omega_m=0.307$, $\Omega_b=0.048$, $\Omega_\Lambda=0.693$ and a
dimensionless Hubble parameter $h=0.6777$ [CITATION PLANCK 2014]


The selected halos were re-simulated at higher resultion by applying a
zoom-in technique with varying physical realism using the moving-mesh AREPO code
that includes gravity, ideal magnetohydrodynamics (MHD), 
phenomenological descriptions for star formation, chemical enrichment
from supernovae and its stellar feedback.  The simulation also follows
the formation and evolution of black holes together with the Active
Galactic Nuclei feedback \citep{arepo} [CITATIONPAKMORE].  


The 30 zoom-in halos have a dark matter particle mass of $\sim 3\times
10^5$\Msun while the barynic mass resolution is $\sim 5\times 10^4$\Msun.
The softening lenght for gravitational force computation for stellar
particles and high-resolution dark matter particles 
is fixed to be 500 $h^{-1}$ pc in comoving coordinates up to $z=1$,
and 396 pc in physical coordinates afterwards.
The gravitational softening lenght for gas cells changes with the mean
cell radius but is limited to be larger than the stellar softening
lenght and 1850 pc physical. 
This setup corresponds to \texttt{Level4} resolution.
From these 30 halos, 6 of them where re-simulated at higher resolution
taking into account a spatial factor of 2 in each spatial dimension,
this is the \texttt{Level3} resolution.  
There are $\sim 4\times 10^6$ dark matter particles within the virial radius
of \texttt{Level4} halos at $z=0$, this number increases to $\sim 3\times 10^7$ in
\texttt{Level3} simulations. 
 
In this work all the results that we report at $z=0$ as a function of radius and
the correlations with disk properties correspond to the 30 halos in
the \texttt{Level4} resolution. 
For the measurement on shape evolution with time we use the 6 halos in
the \texttt{Level3} simulations.
Finally, all  halos described so far have also been simulated at the same
resolutions without MHD using dark matter particles only, we refer to
these halos as the DMO (Dark Matter Only) sample.



\begin{figure*}
  \centering
  \subfloat[DM shape at small radius]{\includegraphics[width=0.5\textwidth]{level4_DM_halo_24_2.png}} 
  \hfill
  \subfloat[DM shape at large radius]{\includegraphics[width=0.5\textwidth]{level4_DM_halo_24_4.png}} 
  \hfill 

  \subfloat[MHD shape at small radius]{\includegraphics[width=0.5\textwidth]{level4_MHD_halo_24_2.png}} 
  \hfill
  \subfloat[MHD shape at large radius]{\includegraphics[width=0.5\textwidth]{level4_MHD_halo_24_4.png}} 
  \hfill 
  \caption{DM density in a slice with width one tenth of the virial radius.
  The black ellipses show the result of the fitting procedure. 
  Upper panels correspond to DM only simulations, lower panels to MHD simulations.
  Left panels show data at small radii, while right panels at large
  radii.
  The cut is made along the minor axis of the halo shape.}
\label{fig:slices}
\end{figure*}



\begin{figure*}
\centering
{\includegraphics[width=0.6\textwidth]{./pics/MHD_Vs_DM/halo16_new.png}}
\caption{Radial profile for the axial ratios $b/a$, $c/a$, and the triaxiality parameter
  $T$ at redshift $z=0$.
  The continous/dashed lines correspond to the MHD/DM simulations respectively.
  The vertical line corresponds to the virial radius.}
\label{fig:triaxial_radius}
\end{figure*} 


 
\begin{figure*}
\begin{center}
\includegraphics[width=0.9\columnwidth]{./pics/Triaxial_Plane/Lvl_4_Triax_Plane_DM.png}
 \includegraphics[width=0.9\columnwidth]{./pics/Triaxial_Plane/Lvl_4_Triax_Plane_MHD.png}
\end{center}
\caption{Axial ratios for the 30 halos in the Level 3 DM simulations
  on the left and MHD simulations on the right.
   Each dot represents the shape characterization of each halo 
   Red circles represent measurements at the virial radius and the
   blue squares are measurements at 1/8 of the virial radius.   
   The crosses represent the mean values and standard deviation of
   each sample.
   Overall we can see three things. 
   First, in DM only simulations halos are rounder in the outskirts
   than in the inner part.
   Second, halos in MHD are rounder than its DM only counterparts.
   Third, halos in MHD simulations are slightly rounder in the inner
   part compared to the outskirts.}
  \label{fig:triaxiality_plane}
\end{figure*}


\begin{figure*}
\includegraphics[width=0.9\columnwidth]{triaxialiy_distro_DM.pdf}
\includegraphics[width=0.9\columnwidth]{triaxialiy_distro_MHD.pdf}
\caption{Cumulative fraction of the triaxiality for all 30 halos in
  Level 4 simulations. Right/left panel correspond to DM/MHD
  simulations respectively. 
  This quantifies the main trends we find: in DM only simulations
  rounder in the outskirts, halos in MHD are rounder than DM only, and
  in MHD simulations halos tend to be rounder at the inner regions
  than its outskirts.}
\label{fig:triax_cumulative}
\end{figure*}


\begin{figure}
\includegraphics[width=0.9\columnwidth]{delta_triaxialiy_distro.pdf}
\caption{Cumulative fraction of the changes in the triaxility betwen
  MHD and DM simulations at different radii. 
  At all radii at least $80\%$ of the halos became rounder in the MHD
  simulations. 
  However, the effect is more pronounced at smaller radii.}
\label{fig:triax_difference}
\end{figure}

\section{Halo Shape Measurement}

%\subsection{The solid ellipsoid method}

The DM halo shape at a fixed radius is an estimate of either
the isopotential or isodensity surfaces.  
Observational inference models usually estimate the 
isopotential contours which are probed by tracers (gas, stars), while
simulations work with the isodensity contours which can be directly
calculated from particle positions.  
Furthermore, the density contours in thin shells are very sensitive to
the presence of small satelites.  
For this reason we measure the shape by taking
volume-enclosed particles, rather than shell-enclosed.  

We measure the shape using the reduced inertia tensor \citep{Allgood_et_al._2006}, 


\begin{equation}
I_{ij} = \sum_k \frac{x_k^{(i)}x_k^{(j)}}{d^2_k},
\label{eq:inertia}
\end{equation}

where the particle positions are measured from the minimum of the
gravitational potential in each halo and each is weighted by the k-th
particle distance 
$d_k^2=x_k^2+y_k^2+z_k^2$, 

The diagonalization of this tensor yields the eigenvectors and
eigenvalues that represent an ellipsoidal dark matter halo.
The axis lenghts of this ellipsoid $a\geq b \geq c$ are the square
root of the ${\bf I}$ eigenvalues and the direction of the principal
axis are the corresponding eigenvectors 

We start the calculations taking into account particles within a
sphere of radius $R$ and then recharacterize the triaxial parameters
by taking into account particles within an ellipsoid of semi-axes
$r,r/q,r/s$ and re-scaled distance $d^2=x^2+(y/q)^2+(z/s)^2$, where $q
= b/a$ and $s=c/a$ are the previously calculated axial ratios. 
We repeat this process until the average deviation of semi-axes is
less than $10^{-6}$.  
This is the same method used to estimate the halo shape in the DM-only
Aquarius simulations \citep{Vera-Ciro_et_al._2011}. 

Following the convergence critery by \cite{Vera-Crito_et_al.2011} we
restrict the sampling of the ellipsoidal parameters to radii  between
$\sim 2$kpc and $R_{200}$, where  $R_{200}$ correspond to the  radius
enclosing a sphere with 200 times the average dark matter density of
the Universe.  
On average, over the 30 halos in the \texttt{Level4} sample
$R_{200}=200\pm 10$kpc. 
For \texttt{Level3} halos we go down to distances $\sim 0.2$ kpc.  


\section{Results}

\subsection{The effect of baryons at $z=0$}

In the DMO sample we find that halos are rounder with increasing
radius
The upper panels in Figure \ref{fig:slices} illustrates this effect.
The contours show a projected DM slice while the ellipsoid corresponds
to the full 3D shape determination. 
There we see a highly ellipsoidal halo shape at radii $\sim 3$kpc
that becomes less triaxial at $\sim 50$ kpc.


We summarize this trends in Figure \label{fig:triaxiality_plane} by
plotting the results of all the 30 halos in the DMO sample.
There we plot the results for every halo in the $c/a$-$b/a$ plane at
two different radii $R_{200}/8 (\sim 20$kpc$)$ 

our results for all the 30 halos in the DMO sample 


confirming results already reported in the
literature \citep{Vera-Ciro_et_al._2011}. 


We quantify this effect by plotting the axial ratios $q$, $s$,
$s/q$ and the triaxiality $T\equiv \frac{1-q}{1-s}$ as a
function of radius. 
As a representative sample of this relations we show on Figure
\ref{fig:triaxial_radius} the radial trends for Halo 16 at
the highest resolution both for DM and MHD simulations.
The continous lines correspond to the DM simulation and shows how
the inner part of the halo $r\approx 1$\kpch has values of 
$q=0.5$, $s=0.5$ while at the virial radius the same
quantities increase to $q=0.85$, $s=0.6$; in turn the triaxiality
decreases from $T\approx 1.0$ to $T\approx0.4$. 

In the same Figure \ref{fig:triaxial_radius} we also find one of the
main results of our study: halos in MHD simulations are systematically rounder, at
every radius, than its DM-only counterparts.  
The dashed line in Figure \ref{fig:triaxial_radius} shows that at
$r\approx 1$\kpch  in the MHD simulation we have $q=0.85, s=0.8$
($q=0.5, s=0.5$ in DM) and at the virial radius $q=0.95, s=0.75$
($q=0.85, s=0.6$ in DM).  
However, the triaxiality trend is not as monotonous for MHD halos is
it is in DM halos. 
The triaxiality goes almost to zero at an intermediate radius,
$r\approx 10$\kpc in the case of Halo 16 in Figure
\ref{fig:triaxial_radius}, to increase again. {\bf pasa lo mismo para
  todos los halos? es lo mismo para baja y alta resolucion de MHD?}
    

Figure \ref{fig:triax_DM_MHD} summarizes this comparizon for all 30 halos
in the Level 3 simulations. 
Measurements at $R_{\rm vir}/16$ are represented by circles while squares 
are measurements at $R_{\rm vir}$. 
All halos, except two, show that outer regions rounder than inner
regions.
The exceptions are cases where a merging substructures near  the
halo's center gives rise to perturbed axial ratio measurements. 
The right panel on Figure \ref{fig:triax_DM_MHD} shows the same information, this time for
the MHD simulations. 
In this case, the main radial trend continues to hold, albeit less
pronounced.
We also find that halos in MHD simulations are in general rounder than
their DM counterparts


This global behaviour can be also illustrated by the individual case
of Halo 16, shown in Figure \ref{fig:triaxial_radius}.
At the virial radius the halo is rounder than its DM counterpart as
evidenced in the axis ratios $b/a$, $c/a$ and the triaxiality.
In the MHD simulation, going from the virial radius to $10$ \kpch the
$b/a$ ratio is constant almost close to unity while in the DM-only
simulation it decreases to $0.5$.
Between $2$-$10$ \kpc the triaxiality in the DM-only simulation is in
the range $0.9-1.0$, while the MHD simulation has values between
$0-0.5$. 


In Table \ref{table:median_axial_ratio_DM} and Table \ref{table:median_axial_ratio_MHD}
we summarize these trends for the median values of the axis ratio in
the DM and MHD simulations
The lower/upper bounds correspond to the first/third quartiles. 
Table  \ref{table:median_axial_ratio_DM} shows how halos are
monotonically rounder with increasing radius; while the comparison
against Table \ref{table:median_axial_ratio_MHD} shows how DM halos
are consistently rounder in MHD simulations than DM-only runs at every
radius. 





\begin{table}
\setlength{\tabcolsep}{3pt}
\begin{center}
\begin{tabular}{l|ccccc}
 &$R_{1/16}$& $R_{1/8}$& $R_{1/4}$& $R_{1/2}$& $R_1$\\
\hline 
$b/a$ &$0.55^{+0.07}_{-0.07}$&$0.57^{+0.09}_{-0.08}$&$0.61^{+0.15}_{-0.08}$&$0.65^{+0.18}_{-0.10}$&$0.70^{+0.13}_{-0.10}$ \\ [0.1cm]
$c/a$ &$0.42^{+0.12}_{-0.03}$&$0.45^{+0.11}_{-0.04}$&$0.49^{+0.09}_{-0.05}$&$0.52^{+0.10}_{-0.05}$&$0.56^{+0.10}_{-0.05}$\\ [0.1cm]
$T$ &$0.89^{+0.03}_{-0.08}$&$0.88^{+0.04}_{-0.12}$&$0.84^{+0.08}_{-0.23}$&$0.81^{+0.08}_{-0.29}$&$0.75^{+0.14}_{-0.25}$\\ [0.1cm]
\end{tabular}
\end{center}
\caption{Median values of axial ratios $q,s$ and triaxiality parameter
  $T$ for DM halos in DM-only simulations at different radii. The
  median is computed over the 30 halos in Level 3 simulations.}  
\label{table:median_axial_ratio_DM}
\end{table}



\begin{table}
\setlength{\tabcolsep}{3pt}
\begin{center}
\begin{tabular}{l|ccccc}
 &$R_{1/16}$& $R_{1/8}$& $R_{1/4}$& $R_{1/2}$& $R_1$\\
\hline 
$b/a$ &$0.93^{+0.04}_{-0.04}$&$0.95^{+0.03}_{-0.03}$&$0.95^{+0.02}_{-0.05}$&$0.93^{+0.04}_{-0.06}$&$0.93^{+0.04}_{-0.10}$\\[0.1cm]
$c/a$ &$0.73^{+0.05}_{-0.09}$&$0.73^{+0.07}_{-0.10}$&$0.73^{+0.08}_{-0.10}$&$0.73^{+0.09}_{-0.08}$&$0.75^{+0.07}_{-0.11}$\\[0.1cm] 
$T$ &$0.31^{+0.15}_{-0.22}$&$0.20^{+0.24}_{-0.12}$&$0.24^{+0.20}_{-0.12}$&$0.30^{+0.26}_{-0.16}$&$0.36^{+0.23}_{-0.23}$\\[0.1cm] 
\end{tabular}
\end{center}
\caption{Median values of axial ratios $q,s$ and triaxiality parameter
  $T$ for DM halos in MHD-only simulations at different radii. The
  median is computed over the 30 halos in Level 3 simulations.}  
\label{table:median_axial_ratio_MHD}
\end{table}



\subsection{Shape evolution with cosmic time}


\begin{figure*}
  \includegraphics[width=\columnwidth]{./pics/Redshift/halo_27_DM_Z_correlation.png}
  \includegraphics[width=\columnwidth]{./pics/Redshift/halo_24_level3_MHD_Z_Triax.png}
  \caption{Radial profile at $z=0$ (dashed line) and 
    the historical profile at the virial radius (colour line). 
    Results for DM simulations on the left, for MHD on the right.{\bf
      Update with halo 24}}
  \label{fig:redshift_triaxial}
\end{figure*}

 
\begin{figure*}
  \subfloat[halo 16 DM]{\includegraphics[width=\columnwidth]{./pics/Redshift/halo_16_level3_DM_Z.png}}
  \subfloat[halo 16 MHD]{\includegraphics[width=\columnwidth]{./pics/Redshift/halo_16_level3_MHD_Z.png}}
  \caption{Radial profile (comoving) of axial ratios for halo 16 in
    terms of redshift (color). This halo maintains its shape until
    $z\approx 1$ obviating the systematic rounding effect in time from
    asymmetric potentials. Each plot-line represents the radial
    profile at a determined redshift. Conclussion: Mainly for b/a, it increases for:
    (1) for bigger radii (fixed redshift)(2) for lower redshift (fixed
    radius) It explains the correlation between radial profile and
    history, but does not require that curves match in the triaxial
    plane. {\bf En esta grafica el panel de abajo deberia ser la triaxialidad.}}
  \label{fig:RedshiftGood}
\end{figure*}

We know that DM-only have a steady and monotonous
trend towards larger sphericity with increasing radius. 
This radial trend is mimicked when the shape is measured at the virial
radius as a function of cosmic time.
The halo should become rounder with decreasing redshift, this is
expected by the continuous influence of the gravitational potential on
a collisionless fluid \citep{Vera-Ciro_et_al._2011}. 

We show in the left panel of Figure \ref{fig:redshift_triaxial} an
example with this effect in Halo 27 of the DM simulations:
the radial triaxiality at $z=0$ and the historical triaxiality at
$R_{\rm vir}$ are correlated. 
The right panel on the same Figure \ref{fig:redshift_triaxial}
illustrates how this correlation is absent in MHD simulations. 

This means that for DM-only halos one can approximate its shape at
higher redshift by simply sampling its 
current shape at a smaller radius. 
This correlation seems to be prompted by the continous inside-out
build-up of the dark matter halo; baryonic effects seem to jeopardize
the apparent smoother growth in DM-only simulations.

%Discuss if this correlation may be recovered if compared for example at Disk radius in stead of virial radius.\\


\subsection{DM halo - Stellar Disk Alignment}


A common assumption in observational models of the MW DM halo is that
its minor axis is perfectly aligned with the disk axis.
Although it is a reasonable assumption to guarantee the
stability of the galactic disk in simplified models of isolated
galaxies, this might not hold in an explicit cosmological context.

To examine the validity of this assumption  we sampled the shape at 5
different radii and plotted  directions in a cartesian coordinate
system where the $z$-axis always corresponds to the minor axis
measured at the virial radius.

{\bf En este caso hace falta la figura con la distribuci\'on integrada de cos theta}.
Figure \ref{fig:alignment} shows the cumulative distribution of
$\cos\theta$, where $\theta$ is the angle betwen the DM halo minor
axis and the stellar disk angular momentum.  
Each line shows the results at different radii.
We find that the majority of the disks are aligned with the minor axis
of their DM halo within $\approx 30^{\circ}$. 
Furthermore, if there is a good alignment measured at the virial
radius, this alignment is well conserved at smaller radii.

{\bf En este caso hace falta la figura de la evoluci\'on de cos theta
  como funci\'on del redshift}
However, there are some disks that show strong missalignments. 
To better understand the missalignments we plot the evolution of
$\cos\theta$ with time to find that the missalignment has been present
for the last XXX Gyr.
This trend is summarized in Figure \ref{fig:alignment_history}.

%\begin{figure*}
%  \centering
%  \subfloat[Perfectly aligned Axes]{\includegraphics[width=0.6\columnwidth]{./pics/well_axes.png}}
%  \hfill
%  \subfloat[Somewhat aligned Axes]{\includegraphics[width=0.6\columnwidth]{./pics/rotating_axes.png}}
%  \hfill
%  \subfloat[Chaotic Axes]{\includegraphics[width=0.6\columnwidth]{./pics/chaotic_axes.png}}
%  \hfill
%  \caption{Star: Minor Axis
%    Triangle: Medium Axis
%    Square: Major Axis
%    Color: Radii at which shape was sampled (show radii sampled)
%    Contour color: If orientation is above or below in this projection
%    Cross (black): Orientation of the stellar disk
%    Conclusion: Axes (minor) are not (generally) aligned with the
%    stellar disk nor are they usually aligned with each other from
%    different radii. We show some cases that may happen.} 
%  \label{fig:alignment}
%\end{figure*}

\textbf{Discussion about the distribution of alignments and their
  evolution in time: Precesion or temporary instabilities?} 

\section{Discussion}


\subsection{What drives the rounding effect?}

From our characterization of radial shapes it is clear that
MHD halos are rounder than DM halos every sampled radii. 
It is also noticeable that the rounding effect of baryons is stronger
at the disk regime, where the DM halo is almost perfectly oblate. 
Furthermore, MHD halos tend towards more oblate shapes (T < 0.5)
despite DM halos tendency towards more prolate shapes (T>0.5). 
This rounding effect can be explained by gravitational effect of the
flattened axisymmetric galactic disk. 
It also explanes the weakness of this affect around $\approx 100$kpc,
where the disk potential is weaker compared to the DM halo potential. 
  
\begin{figure}
\centering
\includegraphics[width=0.5\textwidth]{cumulative_alignment.pdf}
\caption{Cumulative distribution of alignment $\cos\theta$ between the
  minor axis of the dark matter halo at different radii and the
  stellar disk.} \label{fig:alignment_out}
\end{figure} 

Following that linea reasoning, one would also expect that the rounding
effect of baryons is related to some  galactic parameters such as its
component masses and radii. 
We look for these kind of correlations and to find that the strongest
can be found for the stellar density.
However, the correlation is relatively low ({\bf cuanto vale el
  coeficiente de correlation}) most likely due to the different
formation histories.
In other words, the effect of the baryonic disk on the shape of the DM halo
does not fully explain the deviation from oblateness of MHD halos at
$r<10$kpc. 

{\bf Finalmente cuales son las cantidades disponibles en el disco?
  cuales son las que muestran una mayor correlacion? Haria falta un
  plot con todas las correlaciones y la mencion de alguna prueba de
  machine learning del fit lineal para ver cuales son los parametros
  mas importantes.}
  
  We measured the sphericity of a halo shape defined as the distance 
  from the point that characterizes the shape in the triaxiality plane
  to the point $(q=1,s=1)$, which represents a perfect sphere. Taking this 
  as our metric of interest, the sphericity question becomes a search for
  important variables and tendencies that affect this quantity. For this case,
  given the small sample of 30 halos, we chose to study this problem with different 
  3 methods to account for the lacking statistical consistency. We analyzed the effect
  of the Star disk Radius, Baryonic fraction, Gas density, BH density, Gas Radius and
  Star density.\\
   
  We started with a simple linear Lasso regression to force a classification of 
  important variables. Then, we converted this problem in a binary classification problem
  by partitioning the samples in two balanced groups according to the studied metric which was then fitted using Random forest.
  Finally, we concluded this analysis with an MCMC sampling of the likelyhood of a linear model
  with varying slope and intercept parameters.\\
  
  The results of each method are presented below:
  
  \begin{table}
\setlength{\tabcolsep}{3pt}
\begin{center}
\begin{tabular}{l|ccc}
 Variable/Method& Lasso& Random Forest & MCMC \\
 & (slope) & (feature importance)&(signal-to noise)\\
\hline 
Star disk Radius &$-0.168 (1)$&$23\% (2)$&$-0.140$  (3)\\[0.1cm]
Gas Radius&$0.081 (5)$&$12.6\% (5)$&$0.217 (1)$\\[0.1cm] 
Star Density &$-0.064 (6)$&$15.2\% (4)$&$-0.102 (6)$\\[0.1cm]
BH Density&$0.084 (4)$&$16.1\% (3)$&$0.108 (5)$\\[0.1cm] 
Gas Density &$0.097 (3)$&$28.4 \%(1)$&$0.202 (2)$\\[0.1cm] 
Baryonic fraction &$-0.167 (2)$&$4\% (6)$&$/0.146 (4)$\\[0.1cm]  
\end{tabular}
\end{center}
\caption{Results of variable importance of the three methods used to analyse
		 the effect of baryons on the rounding effect of halos. Each method has a different
		 metric for the variable importance and are not comparable but in rankings}  
\label{table:sphericity}
\end{table} 

\begin{figure*}
\centering
\includegraphics[width=0.9\textwidth]{delta_T_correlations.pdf}
\caption{Changes in the triaxiality at $R_{\rm vir}/8$ as a function of different halo and disk properties.}
\label{fig:sphericity}
\end{figure*}    

\textbf{Actually we have not examined the relation of c/a in MHD halos
  with some measure of c/a from the disk, that is something like
  Zdisk/Rdisk. This actually would make more sense from a physical
  point of view: effect of the potential. } 


\textit{ Talk about source of triaxiality at the inner parts of the
  halos (bar?). This source of triaxiality at the inner parts explains
  why the axial ratios are $\approx 0.95$ and not exactly $1$. We
  should also discuss that the decrease in the axial ratios for bigger
  radii may actually be bigger/steeper but it is dimmed by the
  contribution of inner parts.} 


%\begin{figure}
%  \includegraphics[width=\columnwidth]{./pics/Delta asphericity Vs Star Density.png}
%  \caption{Difference in asphericities between MHD and DM shapes Vs
%    Star Density of the simulation. Unsure about this graphic. Take
%    delta asphericity as the strength of the rounding effect of
%    baryons.}  
%  \label{fig:Star_Density_effect}
%\end{figure}



\begin{figure*}
\centering
\includegraphics[width=0.4\textwidth]{q_pot_vs_q_den.pdf}
\includegraphics[width=0.4\textwidth]{s_pot_vs_s_den.pdf}
\caption{Comparison of $q$ and $s$ calculated with isopotential or
  density enclosed volumes. 
  Continuous lines represent the Binney and Tremaine approximations
  and the dots are the measurements on the simulations.}
\label{fig:shape_comp} 
\end{figure*} 


\subsection{Comparison against observational constraints}



\begin{table*}
\begin{tabular}{|l|cc|c|c|p{7cm}|}\hline
Reference&$q_{\rho}$&$s_{\rho}$&$R$&$\theta$&
Methodology\\ 
\hline \hline \citet{Olling_and_Merrifield_2000}& $1.00$
& $0.80$ &  $\simeq 8$kpc & $0^{\circ}$&Stellar dynamics and HI
density. \\

\multirow{2}{*}{\citet{Banerjee_and_Chanda_2011}}&${1}$&${1}$&$9$kpc&$0^{\circ}$&Method:
HI
gas. \\ &${0.5}$&${0.5}$&$24$kpc&$0^{\circ}$&Monotonical
change between radial regimes.\\

\citet{Loebman_et_al._2012}&${1.00}$&${0.47}$&$\sim
20$kpc &$0^{\circ}$&Method: SDSS statistics\\

\cite{Johnston_et_al._2005}&${1}$&$0.83-0.92$&$\lesssim
60$kpc&$0^{\circ}$&Method: Sagittarius stream\\

\citet{Bovy_et_el._2016}&${0.95}$&${0.95}$&$\lesssim
20$kpc&$90^{\circ}$ & Stellar streams\\

\citet{Deg_and_Widrow_2013}&$0.72$&$0.28$&
$20$kpc-$60$kpc$ $&$90^{\circ}$&
Mid-axis orientation. Sagittarius stream\\\hline\hline

\citet{Abadi_et_al._2010} &${0.98}$&${0.85}$& & & Simulations. Almost
  independent of radius. No feedback: boundary case\\\hline

\end{tabular}
\caption{{\bf Cuales son las incertidumbres? Con respecto a que eje
    esta medido theta?}}
\end{table*}


\begin{table*}
\begin{tabular}{|l|cc|c|c|p{7cm}|}\hline
Reference&$q_{\phi}$&$s_{\phi}$&$R$&$\theta$&
Methodology\\ \hline \hline 

\citet{Bowden_et_al._2016}& 0.5-0.66 &0.5-0.66& 5kpc - 10 kpc& $90^{\circ}$&
Weak constraint on prolate halo. SDSS stars dynamics.\\

\multirow{2}{*}{\citet{Vera-Ciro_and_Helmi_2013}}&${1.00}$&${0.90}$&$\lesssim
10$kpc&$0^{\circ}$ & Sagittarius stream \& LMC \\
&${0.90}$&${0.80}$&$\gtrsim 10$kpc&$90^{\circ}$&
Mid-axis orientation on the outside. \\

\citet{Law_and_Majewski_2009}&${0.83}$&${0.67}$&
$\lesssim 60$kpc&$90^{\circ}$&Mid-axis orientation. Sagittarius
stream\\

\citet{Law_and_Majewski_2010}&${0.99}$&${0.72}$&
$20$kpc-$60$kpc$$&$90^{\circ}$&Mid-axis orientation, Sagittarius
stream\\

\citet{Deg_and_Widrow_2013} & $0.82$&$0.40$&
$20$kpc $60$kpc$ $&$90^{\circ}$&
Mid-axis orientation. Sagittarius stream\\\hline \hline

\citet{Chua_et_al._2018}&${0.88\pm0.10}$&${0.70\pm0.11}$&$0.15R_{200}$&
& Illustris\\

\citet{Bryan_et_al._2013}&$0.84-0.86$&$0.66-0.70$&$R_{200}$& & 
For different cosmologies and feedback recipies. Calculated from a fit
at $M_\odot=10^12$\\\hline 

\end{tabular}
\caption{{\bf Cuales son las incertidumbres? Con respecto a que eje
    esta medido theta?}}
\end{table*}
%\end{multicols}



Half of the observational constraints are computed in terms of
isodensity contours, the other half in terms of isopotential contours.
To compare our results against the second kind we must either
translate the isodensity results or recompute in terms of isopotential
regions.
For this purpose, we run a simple iterative algorithm to find an
approximation of the shape of the isopotential contour. 

First, we calculate the mean and standard deviation of the potential over a
spherical shell of width equals to $10\%$ of the radius at which it is
sampled. 
Then, we calculate the inertia tensor of particles with potential
within $1\sigma$ around the mean potential and calculate its triaxial
characterization with the reduced inertia tensor. 
We repeat the
process of calculating the potential mean and standard deviation until
convergence is achieved with tolerance of $10^{-4}$. 

In Figure \ref{fig:density_potential} we compare the two approaches to
measure the axial rations and plot the analytic expectation for $q$, 
$(1-q_{\phi})\approx \frac{1}{3}(1-q_{\rho})$
\citep{Binney_and_Tremaine_2008}, taking the volume-enclosed axial
ratios  as an approximation for the isodensity contour ratios
$q_{\rho}$. 
This plot is computed at four different radii.

Although the analytic expression is meant to be used on the outer
regions of logarithmic axisymmetric halos, it works well as a first
approximation for nearly axisymmetric halos as those produced by our
simulations 
We find that the difference between the measured and the
approximated isopotential axial ratios is not bigger than $quantity
percent$ {\bf }. 


{\bf Aqui hace falta una grafica comparando explicitamente los
  valores para q y para s, esto para diferentes radios.
  Otro punto importante es que se tiene q, pero como se
  relaciona s?
  Ademas hay que calcularlos para los radius aproximados que se
  presentan en las tablas observacionales y en las de simulaciones.
  10-20kpc se acerca a 1/8Rvir, 20-25 1/4Rvir, 60kpc aproximadamente 1/2Rvir
}


\begin{table}
\setlength{\tabcolsep}{3pt}
\begin{tabular}{l|cccc}
 & $R_{1/8}$& $R_{1/4}$& $R_{1/2}$& $R_1$ \\
\hline \hline
$\bar{q}$&$0.98^{+0.01}_{-0.02}$&$0.97^{+0.01}_{-0.04}$&$0.96^{+0.03}_{-0.06}$&$0.94^{+0.03}_{-0.07}$ \\[0.1cm]
$\bar{s}$&$0.89^{+0.04}_{-0.06}$&$0.88^{+0.04}_{-0.04}$&$0.87^{+0.05}_{-0.05}$&$0.85^{+0.05}_{-0.05}$ \\[0.1cm]
$\bar{T}$&$0.18^{+0.23}_{-0.10}$&$0.36^{+0.19}_{-0.21}$&$0.40^{+0.26}_{-0.20}$&$0.48^{+0.23}_{-0.21}$ \\[0.1cm]
\hline
\end{tabular}
\caption{Median values of isopotential axial ratios $q,s$ and triaxiality parameter $T$ for DM halos in MHD simulations at different radii (columns). }
\label{tabe:isopotential}
\end{table}


\section{Conclusions}


\section*{Acknowledgements}
This project has received funding from the European Union's Horizon
2020 Research and Innovation Programme under the Marie
Sk\l{}odowska-Curie grant agreement No 734374. 

 \bibliographystyle{mnras}
 \bibliography{references}
\end{document}

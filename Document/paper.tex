\documentclass[a4paper,fleqn,usenatbib]{mnras}
\usepackage[T1]{fontenc}
\usepackage{ae,aecompl}


\usepackage{graphicx}	% Including figure files
\usepackage{amsmath}	% Advanced maths commands
\usepackage{amssymb}	% Extra maths symbols
\usepackage{subfig}
\usepackage{array}


\usepackage{multirow}
\usepackage{multicol}
\usepackage{blindtext}
\newcolumntype{?}{!{\vrule width 1pt}}
\newcommand{\Msun}{\,{\rm Mpc}$_{\odot}$\,}
\newcommand{\Mpch}{\,{\rm Mpc}\,\ifmmode h^{-1}\else $h^{-1}$\fi}
\newcommand{\kpch}{\,{\rm kpc}\,\ifmmode h^{-1}\else $h^{-1}$\fi}
\newcommand{\kpc}{\,{\rm kpc}\,}


\newenvironment{Table}
   {\par\bigskip\noindent\minipage{\columnwidth}\centering}
   {\endminipage\par\bigskip}

\title[The shape of dark matter haloes at $z=0$ in Auriga]
{Dark matter halo shape at $z=0$ in the Auriga simulations: radial
  evolution and alignment with the stellar disk}
\author[Jesus Prada,  Jaime E. Forero-Romero, Volker Springel ]{
Jesus Prada,$^{1}$\thanks{E-mail: jd.prada1760@uniandes.edu.co}
Jaime E. Forero-Romero,$^{1}$
Volker Springel$^{2}$
\\
% List of institutions
$^{1}$Departamento de F\'sica, Universidad de los Andes, Cra. 1 No.
18A-10, Edificio Ip, Bogot\'a, Colombia.\\
$^{2}$Heidelberg Institute for Theoretical Studies,
Schloss-Wolfsbrunnenweg 35, D-69118 Heidelberg, Germany.\\
}

% These dates will be filled out by the publisher
\date{Accepted XXX. Received YYY; in original form ZZZ}

% Enter the current year, for the copyright statements etc.
\pubyear{2019}

% Don't change these lines
\begin{document}
\label{firstpage}
\pagerange{\pageref{firstpage}--\pageref{lastpage}}
\maketitle

% Abstract of the paper
\begin{abstract}
We present shape measurements of dark matter halos at redshift $z=0$
in a suite of 30 cosmological simulations from the Auriga project. 
We compare the results in full magnetohydrodynamics against dark
matter only physics. 
We find a strong influence of baryons in making dark matter halos
rounder at all radii compared to its dark matter only counterparts.
This change in triaxility is more pronounced towards the inner
regions of the halo.
In simulations with baryons we measure a twist in the dark matter
shape as a function of radial distance; this effect is almost absent
in dark matter only simulations.
We quantify this twisting effect as a changing degree of alignment
bewteen the halo shape with the stellar disk shape as a function of radius.
An almost perfect alignment between the two components appears at
$0.25R_{200}\sim56$ kpc.  
We show that the twist in the radial shape evolution correlates
positively with triaxility.
Furthermore, at distances $\lesssim 30$ kpc, rounder and coherent dark
matter distributions correlate with extended massive stellar disks and
low gas densities. 
In a comparison against observational constraints of dark matter halo
triaxiality for the Milky Way we find that $20\%$ of halos our sample
are consistent with observational results derived from the Pal 5 stream
that favor $b/a\approx 1$ and $c/a>0.77$. Including baryons is a
required element to achieve this level of agreement.
In contrast, all simulations (dark matter only and with baryons) are at
odds with the constrains derived from the Saggitarius stream that
favor $b/a\approx 0.97$ and $c/a\approx0.44$.
\end{abstract}

% Select between one and six entries from the list of approved keywords.
% Don't make up new ones.
\begin{keywords}
galaxies: evolution --- galaxies: formation --- galaxies: haloes ---
dark matter
\end{keywords}

%%%%%%%%%%%%%%%%%%%%%%%%%%%%%%%%%%%%%%%%%%%%%%%%%%

%%%%%%%%%%%%%%%%% BODY OF PAPER %%%%%%%%%%%%%%%%%%

\section{Introduction}

Accurate observations and modeling of our Galaxy 
shape our understanding of the Universe as a whole.  
Explaining the Galaxy's matter composition and kinematical state is 
equivalent to figuring out its formation process in a cosmological
context. 
As a first approximation, the Milky Way's (MW) morphology and
separation into global kinematically coherent components, such as its
disk and bulge, can be used to support the  existence of a Dark Matter
(DM) component to explain its dynamics around the solar neighborhood
\citep{2000MNRAS.311..361O,2009PASJ...61..227S,2010JCAP...08..004C,2013ApJ...779..115B,Iocco15}.  

Improvements in the sensitivity of astronomical observations have 
extended this kind of modelling to outter regions of our Galaxy by measuring
the stellar streams resulting from infalling globular clusters or
satellite galaxies that got tidaly disrutpted by the gravitational
potential of the Milky Way
\citep{1998ApJ...495..297J,1999MNRAS.307..495H, 1999MNRAS.307..877T}. 
The interpretation of such fossil records requires the the
determination of the full three-dimensional gravitational potential of
the Milky  Way.

The observational constraints on the gravitational potential shape can then
be confronted against the expectations from different galaxy formation
models in an explicit cosmological context.
For instance, in the current dominant paradigm of Cold Dark Matter
(CDM) dominated Universe galaxies are expected to be hosted by
triaxial DM halos. To what extent the CDM expectations are born out by
observations in our Galaxy? Can the MW's DM halo shape be considered
typical or atypical in a cosmological context?

This ellipsoidal shape is mostly due to the anisotropical and
clumpy accretion of matter influenced by environmental structures.
Numerical studies how that the shape has a strong mass dependence
\citep{Allgood_et_al._2006}, halos are also rounder at the outerskirts
than at the inner part. 
Shape also evolves with cosmic time, halos get
rounder as they evolve.  


\citep{Dubinski91}


Observationally some studies prefer oblate (i.e. a=b>c) configurations at small
distances around $\leq 20$ kpc
\citep[see][]{Law_and_Majewski_2010,Bovy_et_el._2016,Loebman_et_al._2012,Olling_and_Merrifield_2000,Banerjee_and_Chanda_2011} 
and more triaxial and prolate configurations on the outter distances
$\geq 20$ kpc 
\citep[see][]{Vera-Ciro_and_Helmi_2013,Law_and_Majewski_2009,Deg_and_Widrow_2013,Banerjee_and_Chanda_2011}.
However, some  studies are inclined towards prolate configurations even at the inner
parts of the halo \citep[see][]{Bowden_et_al._2016}, and
although it previously seemed that a triaxial DM halo on the
outerskirts would be necessary to fully explain the characterization
of the Sagittarius stream \citep{Law_and_Majewski_2009}, recent studies
questioned this claim by reporting inconsistencies with narrow stellar
streams \citet{Pearson_et_al._2015} or finding that
the relaxation of other constraints may make this claim unnecessary
\citet{Ibata_et_al._2013}. 

In simulations there is strong evidence claiming that the presence of
baryons produces axisymmetrical halos.  
For instance, some studies have shown that the DM halo shape must be
axisymmetrical to ensure the stability of a hydrodynamical disk
embeded in a static DM halo. 
Other have studied this rounding effect by simulating the disk as rigid
potential inside an N-body triaxial DM
halo \cite{Debattista_et_al._2008,Debattista_et_al._2013,Kazantzidis_et_al._2010}
finding that the halo responds to the disk by becoming less triaxial. 

Rounding effect of baryons
\citep{Dubinski94}

The caveat of the studies mentioned above is that they do not
follow baryons in the whole cosmological context. 
Other studies overcome this limitation by using resimulations 
\citep{Abadi_et_al._2010,Bryan_et_al._2013} finding that the
feeback related to star formation in the disk drives the strenght of
the round effect. 

Abadi2010 -> well aligned countours.

Recently \cite{2018arXiv180907255C} made a study in a cosmological
simulation to compare the effect of including baryons. They do find,
on average, rounder halo shapes once hydrodynamic effects are
included, but it is uncertain the strenght of this statistical effect
on galaxies similar to the MW.


\citep{JingSuto02} report the twisting in their simulations.
They also find a universal radius where the alignment of the internal 
region (defined as the region with 2500 times the critical density) 
is perfect with the shell under consideration.
They blame it on high c/b and undecided vectors.
but in our case c/b increases monotonically with radius, and the
alignment with the disk is not monotonous. In our case this cannot be
the only explanation as the triaxiliaty correlation with the twist 
strongly depends on the radius.

\citep{Pedrosa10}


All these difficulties (enough numerical resolution, explicit
cosmological context, appropriate feedback physics to produce
realistic MW disks) have limited the studies that want to study the
rounding effect of baryons in MW-like galaxies.
In this work we overcome all these limitations by analyzing the
results of state-of-the-art hydrodynamical simulations of isolated
halos that resemble the Milky Way.
We also perform a convergence study with simulation performed at
different resolution levels and explicitly compare the role of DM only
vs. DM+hydro on the MW DM halo shape.


\section{Numerical Simulations}


The Auriga project offers cosmological zoom in simulations of MW-sized 
dark matter halos in a $\Lambda$CDM cosmology. 
This simulations come in two versions: dark matter only and
baryonic physics including magetohydrodynamics (MHD).
A detailed description of the simulations and their
disk properties can be founc in \citep{auriga}, here we summarize its
main features.

The objects in those simulations were selected from a set of 30
isolated halos in the Evolution and Assembly of GaLaxies and their
Environments (EAGLE)  project \citep{Eagle}.   
These halos were randomly selected from a sample of the most isolated
halos at $z=0$ whose virial mass $M_{200}$ was between $10^{12}M_\odot$ and
$2\times 10^{12}M_\odot$. 
The cosmological parameters in these simulatins correspond to
$\Omega_m=0.307$, $\Omega_b=0.048$, $\Omega_\Lambda=0.693$ and a
dimensionless Hubble parameter $h=0.6777$ [CITATION PLANCK 2014]


The selected halos were re-simulated at higher resultion by applying a
zoom-in technique with varying physical realism using the moving-mesh AREPO code
that includes gravity, ideal magnetohydrodynamics (MHD), 
phenomenological descriptions for star formation, chemical enrichment
from supernovae and its stellar feedback.  The simulation also follows
the formation and evolution of black holes together with the Active
Galactic Nuclei feedback \citep{arepo} [CITATIONPAKMORE].  


The 30 zoom-in halos have a dark matter particle mass of $\sim 3\times
10^5$\Msun while the barynic mass resolution is $\sim 5\times 10^4$\Msun.
The softening lenght for gravitational force computation for stellar
particles and high-resolution dark matter particles 
is fixed to be 500 $h^{-1}$ pc in comoving coordinates up to $z=1$,
and 396 pc in physical coordinates afterwards.
The gravitational softening lenght for gas cells changes with the mean
cell radius but is limited to be larger than the stellar softening
lenght and 1850 pc physical. 
This setup corresponds to \texttt{Level4} resolution.
From these 30 halos, 6 of them where re-simulated at higher resolution
taking into account a spatial factor of 2 in each spatial dimension,
this is the \texttt{Level3} resolution.  
There are $\sim 4\times 10^6$ dark matter particles within the virial radius
of \texttt{Level4} halos at $z=0$, this number increases to $\sim 3\times 10^7$ in
\texttt{Level3} simulations. 
 
In this work all the results that we report at $z=0$ as a function of radius and
the correlations with disk properties correspond to the 30 halos in
the \texttt{Level4} resolution. 
For the measurement on shape evolution with time we use the 6 halos in
the \texttt{Level3} simulations.
Finally, all  halos described so far have also been simulated at the same
resolutions without MHD using dark matter particles only, we refer to
these halos as the DMO (Dark Matter Only) sample.



\begin{figure*}
  \centering
  \subfloat[DMO simulation. Shape at small
    radius.]{\includegraphics[width=0.5\textwidth]{level4_DM_halo_24_2.png}}  
  \hfill
  \subfloat[DMO simulation. Shape at large
    radius.]{\includegraphics[width=0.5\textwidth]{level4_DM_halo_24_4.png}}  
  \hfill 

  \subfloat[MHD simulation. Shape at small
    radius.]{\includegraphics[width=0.5\textwidth]{level4_MHD_halo_24_2.png}}  
  \hfill
  \subfloat[MHD simulation. Shape at large
    radius.]{\includegraphics[width=0.5\textwidth]{level4_MHD_halo_24_4.png}}  
  \hfill 
  \caption{DM density in logarithmic scale within a slice of one tenth
    of the virial radius in width. 
    The cut is perpendicular to the short axis of the inertia tensor ellipsoid.
    The black ellipses show the results of the fitting procedure. 
    Upper panels correspond to DMO simulations, lower panels to MHD
    simulations.
    All cases correspond to \texttt{Level4} resolution.
    Left panels show data at small radii, while right panels at large
    radii.    
    This halo showcases the most noticeable effect in all halos
    across the Auriga simulations: DM halos are rounder at all radii
    after baryonic physics is included.}
\label{fig:slices}
\end{figure*}


 
\begin{figure*}
\begin{center}
\subfloat[DMO simulations.]{\includegraphics[width=0.8\columnwidth]{Lvl_4_Triax_Plane_DM.pdf}}
\subfloat[MHD simulations.]{\includegraphics[width=0.8\columnwidth]{Lvl_4_Triax_Plane_MHD.pdf}}
\end{center}
\caption{Axial ratios for all halos in the simulation.
  Left/right panels correspond to DMO/MHD simulations, respectively.
  Triangles (squares) represent the measurements at $R_{200}/16$
  ($R_{200}$) which correspond to physical distances of $14\pm 1$ kpc
  ($230\pm 15$ kpc) respectively.
  Here we can visualize three main trends for the whole halo population.
  First, in DMO simulations halos are rounder in the outskirts
  than in the inner part.
  Second, halos in MHD are rounder than its DMO counterparts.
  Third, halos in MHD are less triaxial in the inner regions than in
  the outskirts.}
  \label{fig:triaxiality_plane}
\end{figure*}


\begin{figure*}
\subfloat[DMO simulations.]{\includegraphics[width=0.8\columnwidth]{triaxialiy_distro_DM.pdf}}
\subfloat[MHD simulations.]{\includegraphics[width=0.8\columnwidth]{triaxialiy_distro_MHD.pdf}}
\caption{Cumulative distribution for the triaxiality at five different radii.
  Right/left panel correspond to DMO/MHD simulations, respectively. 
  In DMO simulations the median triaxiality at all radii is larger
  than $0.5$; only for $20\%$ the triaxility is smaller than $0.5$.
  Furthermore, the triaxility increases as one moves towards the inner
  part of the halo.
  In MHD simulations the situation is reversed.
  The median triaxility at all radii is smaller than $0.5$.
  Moving towards the stellar disk the triaxility decreases towards a median
  value of $T=0.15$.}
\label{fig:triaxial_cumulative}
\end{figure*}


\begin{figure}
\includegraphics[width=0.9\columnwidth]{delta_triaxiality_distro.pdf}
\caption{
  Cumulative distribution for the change in triaxility $\Delta T=T_{\rm
    MHD}-T_{\rm DMO}$ for the same radii used in Figure
  \ref{fig:triaxial_cumulative}. 
  At the virial radius all the halos become less triaxial in the MHD
  simulations. 
  The change in triaxility becomes stronger in the inner regions of
  the dark matter halo.}
\label{fig:delta_triaxial_cumulative}
\end{figure}



%\begin{figure*}
%\begin{center}
%\includegraphics[width=0.35\textwidth]{Z_Triax_level3_set_A_DM.pdf}
%\includegraphics[width=0.35\textwidth]{Z_Triax_level3_set_A_MHD.pdf}
%\end{center}
%\caption{Axial ratios as a function of time (colour circles) and
%  radius (dashed lines). 
%  The left/right columns correspond to DMO/MHD simulations, respectively. 
%  The colour indicates the redshift at which the shape was measured
%  (always at $R_{200}$ at that time.)
%  The radial shapes goes from $R_{200}$ down to $\sim 1$ kpc.
%  These three halos correspond to a first half the sample for the \texttt{Level3}
%  haloes. 
%}
%\label{fig:triaxial_history_A}
%\end{figure*}


%\begin{figure*}
%\begin{center}
%\includegraphics[width=0.35\textwidth]{Z_Triax_level3_set_B_DM.pdf}
%\includegraphics[width=0.35\textwidth]{Z_Triax_level3_set_B_MHD.pdf}
%\end{center}
%\caption{Same as Figure \ref{fig:triaxial_history_A}.
%  These three halos correspond the second half of the sample for the
%  \texttt{Level3} haloes. 
%}
%\label{fig:triaxial_history_B}
%\end{figure*}


\begin{figure*}
\begin{center}
\includegraphics[width=1.0\textwidth]{angles_alignment_DM.pdf}
\includegraphics[width=1.0\textwidth]{angles_alignment_MHD.pdf}
\end{center}
\caption{Angles between the
  principal axis of the halo shape and the principal axis of the
  stellar disk in the MHD simulations at four different radii $\leq 0.5R_{200}$.
  Thin lines correspond to each one of the thirty halos in our sample,
  while the thick line traces the median value.
  Each panel compares the alignment of the corresponding
  major/middle/minor axis both in the halo and the stellar disk.
  In the upper row the haloes come from the DMO simulation, 
  showing that the ellipsoids describing the
  shape are constant as a function of radius for the most part of the
  sample.
  In the lower row the haloes come from the MHD simulation providing a 
  self-consistent comparison with the stellar disks. 
  In this case the dark matter shells twist in all the halos.
  The degree of this twisting is different in each halo.
  Interestingly, an almost perfect halo-disk alignment happens across
  the sample at an intermediate radius of $0.25R_{200}$ ($56\pm 4$kpc).
}
\label{fig:cumulative_alignment}
\end{figure*}


\begin{figure*}
\begin{center}
\includegraphics[width=1.0\textwidth]{correlations_twisting_triaxiality_MHD.pdf}
\end{center}
\caption{Changse of angle alignment of the dark
  matter halo and stellar disk at two different radii ($R_{200}/16$
  and $0.25R_{200}$) as a function of the baryonic disk properties
  already explored in Figure \ref{fig:disk_correlations}.  
  Figure \ref{fig:cumulative_alignment} showed that maximum alignment
  occurs at $0.25R_{200}$ while in the inner regions ($R_{200}/16$)
  considerable missalignments occurr.
 at different radii
 and baryonic disk properties. 
 The label with the $\rho$ value corresponds to the Spearman’s rank correlation coefficient. }
\label{fig:alignment_correlations}
\end{figure*}



\begin{figure*}
\begin{center}
\includegraphics[width=0.8\textwidth]{correlation_T_MHD_disk.pdf}
\end{center}
\caption{Correlations between the halo triaxility at different radii
  and baryonic disk properties. 
  The label with the $\rho$ value corresponds to the Spearman's rank
  correlation coefficient.
  The line is the best linear minimum squares fit.
  The x-axis in the first column is the gas density at the center of
  the galaxy with in a sphere of radius  $1$ kpc \citep{Pakmor17};
  the second column shows the disk to total mass ratio and the last
  column includes the disk optical radius defined to be the radius at which the
  $B$-band surface brightness drops below 25 mag arcsec$^{-2}$ \citep{auriga}.
  The largest correlations are found for the two smaller radii and
  dilute as one approached $R_{200}$.
  Large and massive stellar disks with a low gas content are
  correlated with low dark matter triaxilities.}
\label{fig:disk_correlations}
\end{figure*}


\begin{figure*}
\begin{center}
\includegraphics[width=0.9\textwidth]{correlations_angles_alignment_MHD.pdf}
\end{center}
\caption{Changse of angle alignment of the dark
  matter halo and stellar disk at two different radii ($R_{200}/16$
  and $0.25R_{200}$) as a function of the baryonic disk properties
  already explored in Figure \ref{fig:disk_correlations}.  
  Figure \ref{fig:cumulative_alignment} showed that maximum alignment
  occurs at $0.25R_{200}$ while in the inner regions ($R_{200}/16$)
  considerable missalignments occurr.
 at different radii
 and baryonic disk properties. 
 The label with the $\rho$ value corresponds to the Spearman’s rank correlation coefficient. }
\label{fig:alignment_correlations}
\end{figure*}



\begin{figure}
\begin{center}
\includegraphics[width=0.45\textwidth]{triaxiality_observations.pdf}
\end{center}
\caption{Comparison of our results against other simulations
  by \citet{Chua19} (Illustris) and observational constraints for the 
dark matter halo shape in the Milky Way by \citet{LM10} (LM10) and
\citet{Bovy16} (BBFK16).   
We report our results in the MHD simulations in such a way as to
bracket the radii in the other estimates.
We find that our results are broadly consistent with the Illustris
simulations given the broad dispersion in both data sets.
The consistency with the constraints by \citet{Bovy16} is marginal,
only 1/5 of the halos in our sample seem to be consistent within the
observational reults.
The result by \citet{LM10} would place the Milky Way halo as an
atypical object in the $\Lambda$ CDM context.} 
\label{fig:observations}
\end{figure}


\section{Halo Shape Measurement}


The DM halo shape at a fixed radius is an estimate of either
the isopotential or isodensity surfaces.  
Observational inference models usually estimate the 
isopotential contours which are probed by tracers (gas, stars), while
simulations work with the isodensity contours which can be directly
calculated from particle positions.  
Furthermore, the density contours in thin shells are very sensitive to
the presence of small satelites.  
For this reason we measure the shape by taking
volume-enclosed particles, rather than shell-enclosed.  
This method yields results in good agreement to the isodensity
contours for radii $leq 140$ kpc as explored by
\citep{VeraCiro11}.  


In particular, we measure the shape using the reduced inertia tensor
\citep{Allgood_et_al._2006},  

\begin{equation}
I_{ij} = \sum_k \frac{x_k^{(i)}x_k^{(j)}}{d^2_k},
\label{eq:inertia}
\end{equation}

where the particle positions are measured from the minimum of the
gravitational potential in each halo and each is weighted by the k-th
particle distance 
$d_k^2=x_k^2+y_k^2+z_k^2$.

The diagonalization of this tensor yields the eigenvectors and
eigenvalues that represent an ellipsoidal dark matter halo.
The axis lenghts of this ellipsoid $a\geq b \geq c$ are the square
root of the ${\bf I}$ eigenvalues and the direction of the principal
axis are the corresponding eigenvectors 

We start the calculations taking into account particles within a
sphere of radius $R$ and then recharacterize the triaxial parameters
by taking into account particles within an ellipsoid of semi-axes
$r,r/q,r/s$ and re-scaled distance $d^2=x^2+(y/q)^2+(z/s)^2$, where $q
= b/a$ and $s=c/a$ are the previously calculated axial ratios. 
We repeat this process until the average deviation of semi-axes is
less than $10^{-6}$.  
After convergence we define a unique radius $R$ as the geometrical
mean of the axial lenghts $R=(abc)^{1/3}$.
We use this radial coordinate $R$ to parameterize the spatial changes
in halo shape we report in the following sections.

This is the same method used to estimate the halo shape in the DM-only
Aquarius simulations \citep{VeraCiro11}. 

Following the convergence criterion by \cite{Vera-Crito_et_al.2011} we
restrict the sampling of the ellipsoidal parameters to radii  between
$\sim 2$kpc and $R_{200}$, where  $R_{200}$ correspond to the  radius
enclosing a sphere with 200 times the critical density of the Universe.
On average, over the 30 halos in the \texttt{Level4} sample
$R_{200}=230\pm 15$kpc. 
For \texttt{Level3} halos we go down to distances of $\sim 0.2$ kpc.  

\section{Results}

\subsection{Radial trends at $z=0$}

In the DMO sample we find that halos are rounder with increasing
radius.
The upper panels in Figure \ref{fig:slices} illustrate this effect.
The contours show a projected DM slice while the ellipsoid corresponds
to the full 3D shape determination. 
There we see a highly ellipsoidal halo shape at radii $\sim 3$kpc
that becomes less triaxial at $\sim 50$ kpc.

We summarize this trend in Figure \ref{fig:triaxiality_plane} by
plotting the results of all the 30 halos in the DMO sample.
The left panel shows every halo in the $c/a$-$b/a$ plane at
two different radii $R_{200}/8 (\sim 20$kpc$)$ and $R_{200}$. 
The outer part of the halo is systematically rounder than its inner
region. 
Nevertheless the halo shape can still be considerated to be prolate at
all radii. 
These plots confirm the results already reported in the
literature \citep{VeraCiro11}.

A different picture presents itself in the MHD sample.
There all halos become rounder at all radii than its DMO
counterpart.
The lower panel in Figure \ref{fig:slices} can be directly compared to
its MHD counterpart; there we observe how at large radii the halo
becomes almost spherical. 
The right panel in Figure \ref{fig:triaxiality_plane} shows the
results for the 30 halos in the MHD sample.
This time the bulk of the halos can be considered oblate and close to
spherical. 

In Figure \ref{fig:triaxial_cumulative} we summarize the results at
different radii using the cumulative distributions for the 
triaxility parameter $T$ defined as 
\begin{equation}
T=\frac{a^2-b^2}{a^2-c^2}.
\label{eq:triaxiality}
\end{equation}
The left panel of this Figure shows that in the DMO sample the
triaxility has a median larger than $0.5$ at all radii, furthermore
this median value increases as we move towards the inner part of the
halo.
The right panel shows the exact complementary picture in the MHD
sampe.
There the median triaxility is alwas smaller than $0.5$ and this
triaxility is smaller as we move closer to the galactic disk.


To quantify to what extend the global effect of decreasing
triaxility in MHD simulations compared to the DMO sample 
holds for individual halos. 
We compute $\Delta T\equiv T_{\rm MHD}-T_{\rm DMO}$ the difference between the
triaxility in the MHD and the DMO simulation for each individual halo. 
Figure \label{fig:delta_triaxial_cumulative} shows the cumulative
distribution at the same radii as in
Figure \label{fig:delta_triaxial_cumulative}. 
    

%\subsection{Shape evolution with cosmic time}

%In the previous section we stablished that DMO halos at $z=0$ have
%a steady and monotonous trend towards decreasing triaxility with
%increasing radius.
%\cite{VeraCiro11} found that this  radial trend at $z=0$ 
%correlates with the halo shape as a function of cosmic time, i.e.
%the halo becomes less triaxial as cosmic time progresses. 
%The hypotesis put forward by \cite{VeraCiro11} is that the correlation
%raises as each radial shell retains memory of the accreting conditions
%at the virial radius at the time of collapse.

%Here we use the six \texttt{level3} halos to measure to what extent
%the presence of baryons affect this memory effect.
%In Figures \ref{fig:triaxial_history_A} and
%\ref{fig:triaxial_history_B} we follow the shape evolution measured at
%$R_{200}$ since redshift $z=1$ to the present, these results are
%represened by the coloroured dots.
%We compare this temporal evolution against the radial evolution at
%$z=0$ represented by the dash black lines. 

%The left column of Figure \ref{fig:redshift_triaxial} shows the six
%halos in the triaxility plane. 
%The colour dots show the time evolution while the black line
%corresponds to the radial trend at $z=0$.
%The left/right columns correspond to the DMO/MHD simulations, respectively.


%We start by analyzing to what extent the memory effect is
%already present in the DMO sample. 
%The first feature that jumps out is that \texttt{Au-21} in the DMO
%simulation does not have a simple shape evolution history.
%By visual inspection we find that an extended merging substructure
%intereacted in the main halo showing up in the shape measurements,
%making it more difficult to interpret the memory hypothesis.
%For this reason we exclude this halo from the analysis.

%For the other five halos we find simpler temporal and spatial
%evolution histories. 
%Halo \texttt{Au-6} (Figure
%\ref{fig:triaxial_history_A}) shows a broad
%correlation between the two tracks; close to the cases of halos
%\texttt{Aq-4-C} and \texttt{Aq-4-D} in \cite{VeraCiro11}.
%The four other halos (\texttt{Au-16} in Figure
%\ref{fig:triaxial_history_A}; \texttt{Au-23}, \texttt{Au-24},
%\texttt{Au-27} in Figure \ref{fig:triaxial_history_B}) are four
%cases where the temporal evolution if followed very closely by the
%radial trends.
%We conclude that in the DMO sample 4 out of the 5 high resolution
%halos closely follow the memory hypothesis trends and the remaning
%halo shows a reasonable correlation.

%The situation in the MHD simulations is more diverse.
%As expected by the results in the previous section the broad location
%in the triaxility plane of the temporal and radial evolution in the
%MHD simulation is completeley different from its DMO counterpart. 
%In terms of the correlations expected by the memory hypothesis we see
%two different trends.
%We have the case of Halos \texttt{Au-6}, \texttt{Au-16} and
%\texttt{Au-27} there is still a reasonable correlation between
%temporal and radial evolution. 
%However for halos \texttt{Au-23} and \texttt{Au-24} the radial trends
%go in directions that are not even covered by the temporal evolution.

%Taken at face value, these results show that it is not possible to
%generalize the memory hypotesis to dark matter halos in MHD
%simulations.
%This means that some halos present a amnesia to its formation history,
%more specifically there are cases where the radial shape profile at
%present time does not have any correlation to its evolutionary
%history. 


\subsection{Alignments with the stellar disk}


A common assumption in observational models of the MW DM halo is that
its minor axis is perfectly aligned with the stellar disk minor axis.
Although it is a reasonable assumption to guarantee the stability of
the galactic disk in simplified models of isolated galaxies, this
might not hold in an explicit cosmological context. 
To examine the degree of validity of this assumption we study in this
section the alignment between the eigenvectors of the inertia tensor of
stellar particles within $0.1R_{200}$ ($\sim 23\pm 2$ kpc) and the
eigenvectors of the dark matter halo shape.
All the measurements are done at $z=0$.

In Figure \ref{fig:cumulative_alignment} we summarize our main results
regarding these alignments with the halo shape measured at five
different radii.
The upper row shows the alignment of the halos in the DMO simulations
with the stellar disk in the MHD simulations.
The main objective of this measurement is to calibrate the radial
evolution of the DM halo shape. 
We find that the DM shape remains constant with radius.

The lower row in Figure \ref{fig:cumulative_alignment} shows the
alignments with the halo in the MHD simulations. 
This time the halo shape changes and twists at different radii.
However around the radius of $0.25R_{200}$ there is an alignment
almost perfect between the shapes of the stellar disk and the dark
matter halo. 
Above and below this radius there are halos with a lower degree of
alignment.
Across the three different alignments we measure we verify that the
strongest one is indeed the one between the two minor axis.

Statistically the strongest missalignment is found with the halo shape
at a radius of $R_{200}/16\sim (14\pm1)$ kpc. 
At this radius the mean angle and its standard deviation between the
two minor axis is $18 \pm 21$ degrees, with one extreme case
(\texttt{Au-4}) where the angle is close to $78$ degrees. 
In contrast at $0.25 R_{200}$ the angle between the two axis is
$2\pm 3$ degrees without any extreme outlier.





\section{Discussion}

The first effect that we put in evidence in this paper is the effect
of baryons in producing rounder DM halos.

The strength of the change depends on the numerical resolution, the
method to resolve the hydrodynamics and the models describing star
formation and stellar feedback \citep{Debattista08, Bryan13, Butsky16,
  Chua19, Artale19}.  
The key concept unifying these results is that the baryon distribution
influences and correlates with the dark matter halo shape. 
Here we find that the broad tendency is that massive stellar disks
correlate with spherical dark matter distributions. 

In order to explore this idea in the Auriga simulations we quantify
the correlation between halo shape with baryonic disk properties. 
Looking into the measurements already reported by \cite{auriga} and
\cite{Pakmor17} we find three baryonic quantities that have the
strongest correlation with DM halo triaxiality: the central gas
density in a sphere of radius $1$kpc, the disk to total mass ratio and
the optical radius.

Figure \ref{fig:disk_correlations} shows the correlations of
these quantities with the triaxility at five diferent radii.
We use the Spearmen's rank correlation coefficient to quantify the
correlation strength.
We find that the strongest correlations are found with the halo shape
measured at radii smaller than $0.12R_{200}\sim 28$kpc, which 
is close to the upper limit of the disk optical radius among our
simulation sample. 
The trend is such that halos with large triaxiality correlate with
high gas density and stellar disks with low mass and small size.
In turn, massive and large stellar disks within a low density gas
environment correlate with low halo triaxiliaty. 

Our second results deals with the shape alignment as a function of
radius. 
In concordance with previous results in DM only simulations we find
that the shells of halo shape are well aligned at different radii.
However, in the presences of baryons these shells twist as a function
of radius.
\citep{Debattista08} found a similar effect, but only at radii smaller
than the disk radius. At radii larger than the disk radius, they find,
the dark matter shells stop twisting.
In our case the he twisting effect is robust across our sample of 30
halos at all radii,
its strenght also correlates with the halo triaxiality for radii
smaller than $0.12R_{200}$. 
At larger radii 
Also surprisingly at a radius of $0.25R_{200}\approx 56$ kpc the
alignment between the stellar disk and the dark matter halo is almost
perfect. 

The measurements we have done in the simulations can be compared
against observational results.
We use the observational constraints by \cite{LM10} and \cite{Bovy16}
to place our results in an observational context.
\cite{LM10} used observations of the long Sagittarius tidal stream to
constrain the shape of the gravitational potential.
Their point of depart is that previous studies that assumed an
axisymetric Galactic potential were not able to fit all the available
dynamic constraints for the Sagittarius stream, therefore they
proposed a rigid triaxial potential, with coaxial potential
ellipsoids, for the dark matter component.  
Their results are able to constrain the triaxility of this potential
component, translating these results into a triaxiality of the density
contours they report $(c/a)=0.44$ and $(b/a)=0.97$ at a radius of
$\sim 40$kpc. They do not report any uncertainties for these two
values. 
Looking at their plots of their quality of fit criterion as a function
of dark halo axial scales (Figure 5), we choose a conservative $10\%$
relative uncertainty, close to the uncertainty in those parameters.
The element that made the results by \cite{LM10} (LM10 hereafter) surprising is that
the minor axis of the halo shape is parallel, not perpendicular, to
stellar disk plane. 

The results by \cite{Bovy16} (BBFK16 hereafter) have the same general approach but use
instead the GD-1 \citep{2006ApJ...641L..37G} and Pal 5 \citep{2009AJ....137.3378O}
streams to constraint the shape of the dark matter component of the
galacic halo potential.
They use general models with large degrees of freedom for the galactic
potential in order to measure to what extent these two streams are sensitive
to the triaxiality of the dark matter halo component.
The DM component is written directly as a triaxial density profile
with coaxial ellipsoids and the corresponding potential is found by
numerical integration.
They find that the width of the Pal 5 stream constraints $b/a\approx
1$ and therefore fix it to $b/a=1$, finally they report their most stringent
constrain of $c/a=0.93\pm0.16$ at a radius of $\approx 19$kpc from the
galactic center.

Our Figure \label{fig:observations} shows an explicit comparison in
the $c/a$-$b/a$ plane against the results by LM10 and BBFK16. 
We find $6$ halos with $b/a<0.93$ and $c/a>0.77$ that could be
considered consistent with their shape constraints 
by BBFK16.
In contrast, none of the simulated halos is consistent with the LM10
results. 

The results by LM10 would place then the dark matter halo of our Milky
Way as an extreme outlier in the $\Lambda$ CDM model. 
This extreme prolateness also correlates with the extreme
triaxiality of the 11 classical satellites of the MW ($c/a\approx
0.2$, and $b/a\approx0.9$) with an spatial distribution 
also oriented perpendicular to the MW plane, another highly unusual
feature in the $\Lambda$ CDM model \citep{2018MNRAS.478.5533F}

From the results presented in this work we advance the idea that the
twisting shells in the density is a feature that deserves to be
explored in the process of constraining shape parameters from tidal
stream data. 
The current parameterizations of coaxial density/potential shells
might be too rigid.
The inclusion of a new free parameter describing this degree of
twisting might relax the conflict between the observational
constraints and the numerical results. 

{{\bf alinenaciones y twisting as a function of environment, 
the difference between accretting and stalled halos, }

%Porciani: stalled halos keep their dark matter density profile
%while accreting halos have. Shells don't pile up in accreting halos, 
%they approximately do in stalled halos}

%Porciani: The stalled halos are the most similar to the MW at least in
%terms of the velocity anisotropy parameter an should be embeded in a
%thick filament of tha large scale structure.

%IDEA PARA YEIMY: medir scatter en stellar to DM halo mass ratio as a
%function of the anisotropy parameter or the persistence parameter in 
%Illustris. Primer proxy para el galaxy formation bias. Esto se puede
%sacar pronto. (ademas otra medida de los compacto que es la distribucion
%estelar)
%esto puede ser comparado con los datos de vipers que tiene un sesgo 
%donde se ven diferentes clustering a misma masa para galaxias mas o
%menos concentradas

%otra idea: calcular la entropia del beta skeleton para distinguir 
%el clustering de dos muestras diferentes. i.e. la entropia para
%caracterizar assembly bias.

\section{Conclusions}

In this paper we measured the shape of thirty isolated Milky Way like 
dark matter haloes simulated in the Auriga project using the zoom-in
technique. 
We also used six of these haloes that were simulated at higher
resolution. 
The Auriga project simulated these halos with two different setups:
dark matter only (DMO) simulation and full magnetohydrodynamics (MHD)
includeing star formation and feedback.
We used the shape measurement algorithm by \cite{Allgood06} on the
dark matter halos of these two kinds of simulations.

We find that MHD halos are rounder than DMO halos every sampled radii. 
MHD halos tend towards more oblate shapes (T < 0.5)
despite DM halos tendency towards more prolate shapes (T>0.5). 
It is also noticeable that the rounding effect of baryons is stronger
at the disk regime, where the haloes are close to oblate.

This rounding effect can be explained by gravitational effect of the
flattened axisymmetric galactic disk. 
It also explanes the weakness of this affect around $\approx 100$kpc,
where the disk potential is weaker compared to the DM halo potential. 

The second part of our study on the historical shape finds that in the
presence of baryons the memory effect found by \citep{VeraCiro11} does
not hold for all halos in the high resolution sample.


\section*{Acknowledgements}
This project has received funding from the European Union's Horizon
2020 Research and Innovation Programme under the Marie
Sk\l{}odowska-Curie grant agreement No 734374. 


 \bibliographystyle{mnras}
 \bibliography{references}
\end{document}
